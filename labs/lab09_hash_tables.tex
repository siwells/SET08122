\documentclass[10pt, a4paper, twosize]{article}
%\documentclass[12pt, a4paper, twoside]{book}

\usepackage{helvet}
\usepackage{hyperref}
\usepackage{graphicx}
\usepackage{listings}
\usepackage{textcomp}
\usepackage[
	a4paper,
	outer=2cm,
	inner=4cm,
	top=2cm,
	bottom=2cm
]{geometry}
\usepackage{float}
\usepackage{tabularx}
\usepackage[disable]{todonotes}
\usepackage{color, soul}
\usepackage{amsmath}
\usepackage{algorithmicx}
\usepackage[noend]{algpseudocode}
\usepackage{algorithm}
\usepackage{framed}
\usepackage{subcaption}
\usepackage{titlepic}
\usepackage{fancyhdr}
\usepackage[simplified]{styles/pgf-umlcd}
\usepackage{shorttoc}
\usepackage{url}
\usepackage{paralist}

\definecolor{grey}{rgb}{0.9, 0.9, 0.9}
\definecolor{dkgreen}{rgb}{0,0.6,0}
\definecolor{dkred}{rgb}{0.6,0,0.0}

\lstdefinestyle{DOS}
{
    backgroundcolor=\color{black},
    basicstyle=\scriptsize\color{white}\ttfamily,
    stringstyle=\color{white},
    keywords={}
}

\lstdefinestyle{makefile}
{
    numberblanklines=false,
    language=make,
    tabsize=4,
    keywordstyle=\color{red},
    identifierstyle= %plain identifiers for make
}

\lstset{
  language=Java,                % the language of the code
  basicstyle=\footnotesize\ttfamily,
  numbers=left,                   % where to put the line-numbers
  stepnumber=1,                   % the step between two line-numbers. If it's 1, each line
  numbersep=5pt,                  % how far the line-numbers are from the code
  backgroundcolor=\color{white},      % choose the background color. You must add \usepackage{color}
  showspaces=false,               % show spaces adding particular underscores
  showstringspaces=false,         % underline spaces within strings
  showtabs=false,                 % show tabs within strings adding particular underscores
  frame=single,                   % adds a frame around the code
  rulecolor=\color{black},        % if not set, the frame-color may be changed on line-breaks within not-black text (e.g. comments (green here))
  tabsize=2,                      % sets default tabsize to 2 spaces
  captionpos=b,                   % sets the caption-position to bottom
  breaklines=true,                % sets automatic line breaking
  breakatwhitespace=false,        % sets if automatic breaks should only happen at whitespace
  keywordstyle=\color{blue},          % keyword style
  commentstyle=\color{dkgreen},       % comment style
  stringstyle=\color{dkred},         % string literal style
  columns=fixed,
  extendedchars=true,
  frame=single,
}

%\renewcommand{\chaptername}{Topic}

% New definitions
\algnewcommand\algorithmicswitch{\textbf{switch}}
\algnewcommand\algorithmiccase{\textbf{case}}
\algnewcommand\algorithmicassert{\texttt{assert}}
\algnewcommand\Assert[1]{\State \algorithmicassert(#1)}%
% New "environments"
\algdef{SE}[SWITCH]{Switch}{EndSwitch}[1]{\algorithmicswitch\ #1\ \algorithmicdo}{\algorithmicend\ \algorithmicswitch}%
\algdef{SE}[CASE]{Case}{EndCase}[1]{\algorithmiccase\ #1}{\algorithmicend\ \algorithmiccase}%
\algtext*{EndSwitch}%
\algtext*{EndCase}%

\pagestyle{fancy}
\fancyhf{}
\fancyhead[RO, LE]{\small \rightmark}
\fancyfoot[RO, LE]{\small \thepage}

\begin{document}

%\frontmatter

\begin{titlepage}
\vspace*{5cm}
\begin{center}
\includegraphics[width=.5\textwidth]{images/EdNapUniLogoCMYK}~\\[1cm]

\textsc{\Large Edinburgh Napier University}\\[1.5cm]

\textsc{\LARGE \bfseries SET08122 Algorithms \& Data Structures}\\[0.5cm]

\hrulefill \\[0.4cm]
{\huge \bfseries Lab 9 -- Hash tables \\[0.4cm] }
\hrulefill \\[1.5cm]

\begin{minipage}{0.4\textwidth}
\begin{flushleft} \large
\textbf{Dr Simon Powers} \\
\end{flushleft}
\end{minipage}

\vfill

\end{center}
\end{titlepage}

%\shorttoc{Overview}{0}

%\setcounter{tocdepth}{2}
%\cleardoublepage
%\tableofcontents
%\listoffigures
%\listofalgorithms
%\addtocontents{toc}{~\hfill\textbf{Page}\par}

%\mainmatter

%\input{sections/labs/04_ui}

\section{Aims}
\paragraph{} At the end of this topic you will be able to:

\begin{itemize}
\item Implement an open-addressed hash table.
\item Measure the performance of a hash table.
\item Create a plot of your hash table's performance as its load factor increases.
\item Implement a chained hash table using a linked list.
\end{itemize}

\section{Introducing hash tables}
A common problem in computing is that given a key (e.g. a student matriculation number or car registration plate) we want to retrieve the corresponding data record (e.g. student transcript or car owner details). Hash tables provide one of the most efficient methods of solving this problem. 

The idea is that we will use an array, called a hash table, to store our records. To insert a record into the hash table, we do the following. We take the key, $k$, and put it though a hash function, $h(k)$. The output of this function tells us the index in the array where we should put our record. Then, to later retrieve the record, we just compute $h(k)$ again and this tells us directly the index in the array where the record is located -- no searching required! As such, we can add, remove and retrieve records in \emph{constant time}.

A hash function, $h(k)$, is then a mapping between the set of all possible keys and indices in the array. If the array is the same size as the set of all possible keys then our hash function can guarantee that no two keys will map to the same index. In that case our hash table is said to be directly addressed. But in reality, our array will usually have to be much smaller than the universe of keys. For example, suppose our keys were 8 character names. Using direct addressing, we would need an array of size $26^8$, and the majority would be unused since most character combinations are not names. Consequently, we need to use smaller arrays and handle the case where two or more keys map to the same index, i.e. they \emph{collide}.  


\section{Implementing an open-addressed hash table}
When a collision occurs, we need a way to resolve it and find a place to insert our key. A collision is resolved by \emph{probing} alternate positions in the array until an empty bucket is found. The simplest case is \emph{linear} probing, where if a collision occurs we check indices $h(k)+1$, $h(k)+2$, $h(k)+3$ and so on until an empty is position is found, or we have gone through every position in the array, which indicates that the hash table is full and cannot store any more elements. 

Now that we are moving towards the end of the module, we are going to advance our C programming style to implement this, by building on the basic implementation that you produced in Lab 05. We are going to develop an API for our hash table that supports the following operations:

\begin{itemize}
\item Initialise a hash table.
\item Destroy a hash table to free the memory it occupies.
\item Insert a data element (key, value pair).
\item Search for a data element.
\item Print the keys in the table in the order in which they occur.
\item Print some performance metrics about the table.
\end{itemize}

We will place the prototypes for these API functions in a header file, name \texttt{OAHashTable.h} (which stands for Open Addressed Hash Table). By placing the prototypes in a header file, we only have to include them once even if we want to call the functions from multiple different code files. The code for the \texttt{OAHashTable.h} header file is as follows:

\begin{lstlisting}
#pragma once

struct DataItem 
{
    int data;
    int key;
};

struct OAHashTable
{
    struct DataItem** buckets;
    int size;
    int numInsertions;
    int numSearches;
    int numProbes;
    int numCollisions;
};

struct OAHashTable* initOAHashTable(int);
void printKeys(struct OAHashTable*);
int insert(struct OAHashTable*, int, int);
void destroy(struct OAHashTable*);
void printPerformance(struct OAHashTable *hashTable);
struct DataItem* search(struct OAHashTable*, int);
\end{lstlisting}

As in Lab 05, we have a struct for a data item storing the key and its data. In this simple case the key and data are both integers. In a more realistic application the key might be alpha numeric, in which case we would have to convert it to an integer somehow before passing it to our hash function $h(k)$, and how we do this conversion can be critical to the performance of insert and search operations. The data would typically be a \texttt{void*}, i.e. a void pointer. In other words, a pointer to any type of data, such as another struct. This would allow our implementation of the hash table to be reused for different data types (like templates in C++ or generics in C\# or Java). Here, though, we will stick with ints for simplicity.

The second struct, \texttt{OAHashTable}, encapsulates our hash table (an array of \texttt{DataItems} to be allocated dynamically on the heap) along with other useful data about the table, such as its size, and some other variables to store performance metrics. Finally, the header file contains the function prototypes for our API. Note that we pass a pointer to the OAHashTable that we wish to operate on in as parameters to all of these functions (except for the initialisation function). You have seen this before with the other APIs that you have developed. 

The implementation of the functions goes in a separate .c file, called \texttt{OAHashTable.c}. This separates out the \emph{specification} of the API, which goes in the header file and which is shared with users in plain text, from the \emph{implementation} whose details are irrelevant to users of the API (and which would typically be distributed as a pre-compiled library file).

The function to initialise an open-addressed hash table is as follows:
\begin{lstlisting}
struct OAHashTable* initOAHashTable(int size)
 {
 	struct OAHashTable *hashTable = (struct OAHashTable*)malloc(sizeof(struct OAHashTable));
	// Use calloc to initialise the buckets to NULL, so that we can test
	// for empty buckets.
	hashTable->buckets = (struct DataItem**)calloc(size, sizeof(struct DataItem*));
 	hashTable->size = size;
	hashTable->numCollisions = 0;
	hashTable->numProbes = 0;
	hashTable->numInsertions = 0;
	hashTable->numSearches = 0;
	return hashTable;
 }
\end{lstlisting}
This function initialises a hashTable with a capacity to store \textit{size} data items. These will be stored in an array called \texttt{buckets}. We also initialise some counters for the frequencies of operations to 0, which we will use to compute performance statistics.

The function to insert a data item is similar to that in Unit 05, and looks like this:
\begin{lstlisting}
int insert(struct OAHashTable *hashTable, int key, int data)
{
    struct DataItem *item = (struct DataItem*) malloc(sizeof(struct DataItem));
    item -> data = data;
    item -> key = key;

    int hashIndex = hashCode(key, hashTable->size);
    int numTries = 0;
    if(hashTable->buckets[hashIndex] != NULL)
    {
        hashTable->numCollisions++;
    }
    while(hashTable->buckets[hashIndex] != NULL && numTries < hashTable->size)
    {
        numTries++;
        ++hashIndex;
        hashIndex %= hashTable->size;
    }
    if(numTries == hashTable->size)
    {
        return -1;
    }
    else
    {
        hashTable->buckets[hashIndex] = item;
    }
    hashTable->numInsertions++;
    return 0;
}
\end{lstlisting}

Here, we first look up the hash code, $h(k)$ for our key $k$. A common hash function is $h(k) = k \: \mathrm{mod} \: m$, where $m$ is the size of the table. This function ensures that every possible key maps to an index in the range of the array. We will use this function -- you should code it in a function called \texttt{hashCode}. The condition of the while loop detects if a collision has occurred, i.e. is there already another data item in this position in the array. If yes, we add one to the array index (wrapping around back to the start if necessary), until we either find an empty position, or if we have tried every position in the table (in which case we return -1 as an error code). Adding one each time to the array index to find the empty position closest to the index we hashed to is known as \emph{linear probing}. Ideally we would have no collisions, as they slow down insertion (and searching) performance. But given that collisions are inevitable for large key sets and finite arrays, we want to resolve collisions in as few probes as possible. Note that we record the number of collisions that occur in our \texttt{OAHashTable} struct, since this is an important performance metric.

The function to search for a data item given its key is very similar:
\begin{lstlisting}
struct DataItem* search(struct OAHashTable* hashTable, int key)
{
    hashTable->numSearches++;
    int hashIndex = hashCode(key,hashTable->size);
    int numTries = 0;
    while(hashTable->buckets[hashIndex] != NULL && numTries < hashTable->size)
    {
        numTries++;
        hashTable->numProbes++;
        if(hashTable->buckets[hashIndex]->key == key)
            return hashTable->buckets[hashIndex];
        
        ++hashIndex;
        hashIndex %= hashTable->size;
    }
    
    return NULL;
}
\end{lstlisting}
This time we record the number of probes that occur when searching for an item, as this will be our key performance metric.

We print out the contents of the table as follows:
\begin{lstlisting}
void printKeys(struct OAHashTable* hashTable)
{
    printf("[");
    for(int i=0; i<hashTable->size -1; i++)
    {
        if(hashTable->buckets[i] != NULL)
        {
            printf("%d, ", hashTable->buckets[i]->key);
        }
        else
        {
            printf("-, "); // Indicates an empty space in the table
        }
    }
    // Print the last element without a comma and space, just to be neat
    
    if(hashTable->buckets[hashTable->size-1] != NULL)
    {
        printf("%d", hashTable->buckets[hashTable->size-1]->key);
    }
    else
    {
        printf("-"); 
    }
    printf("]\n");
    
}
\end{lstlisting}

Finally, we need a function to destroy the table when we are finished with it, freeing up all of the memory that we have malloced (or calloced):
\begin{lstlisting}
void destroy(struct OAHashTable *hashTable)
{
	for(int i=0; i<hashTable->size; i++)
	{
		if(hashTable->buckets[i] != NULL)
		{
			free(hashTable->buckets[i]);
		}
 	 }
 	 free(hashTable);
 }
\end{lstlisting}
A user of our API can now free all of the heap allocated memory just by calling our destroy function (you would write a destructor for this in C++, and in C\# and Java the garbage collector would take care of it for you).

The full listing for the \texttt{OAHashTable.c} file is shown at the end of this document.

\textbf{Challenge:} add a remove function to your hash table. The trick with removing from open-addressed hash tables is that we cannot just overwrite a removed data item with NULL. This is because our search functions tops searching when it finds a position with NULL in it. But if we have removed an item, then the item that it is looking for may be in a cluster that continues after the NULL item, so we don't necessarily want to stop the search. When we remove an item, we should replace it with a special flag, and modify our search method to continue to the next probe if it lands on this flag. We can use a \texttt{DataElement} with a key of -1 and data of -1 for this purpose. Create a remove function (which takes in a hash table and a key as a parameter), and modify your search method to check for the flag.

\section{Activity 1: Experimenting with your open-addressed hash table}
Write a main method (in a file called \texttt{main.c}) to initialise a hash table of size 11. Insert the following keys, \emph{in order}, into your hash table (you can use random values for the data items as we are only interested in the distribution of keys for our performance analysis):

6, 2, 13, 5, 17, 39, 1, 12

After inserting the elements, , print the contents of the array to the screen. It should look like this: [-, 1, 2, 13, 12, 5, 6, 17, 39, -, -], where $-$ marks an empty bucket in the array. Search for the elements 6 and 39, which are in the table, and 100, which is not.
 
Notice the phenomenon of \emph{primary clustering}. As collisions start to occur, we get long chains of numbers out of their position, which results in increased time to insert and retrieve items. This is a problem with linear probing.

Now to investigate performance more precisely, create a hash table of size 10000. Generate 1000 random keys and store them in an array. Then insert them into your hash table. After that, search for all 1000 of these keys. Measure the performance using the following function (declared in \texttt{OAHashTable.h} and implemented in \texttt{OAHashTable.c}):
\begin{lstlisting}
void printPerformance(struct OAHashTable *hashTable)
{
    printf("Number of insertions: %d\n", hashTable->numInsertions);
    printf("Number of collisions: %d\n", hashTable->numCollisions);
    printf("Load factor: %.3f\n", (double)hashTable->numInsertions/hashTable->size);
    printf("%% of insertions that collided: %.1f\n", (double)hashTable->numCollisions/hashTable->numInsertions*100);
    printf("Average number of probes per search: %f\n", (double)hashTable->numProbes/hashTable->numSearches);
}
\end{lstlisting}
The load factor tells us how full our hash table is -- for an open-addressed has table it cannot be greater than 1, as a bucket cannot store more than one data item. We are interested in the relationship between the load factor and the average number of probes per search. To ascertain this, increase the load factor by inserting between 1000 to 10000 keys into the table, in increments of 1000, and record the performance after searching. You should plot a graph of the load factor against the average number of probes when searching. At what point do you think it would be worth making the hash table bigger to increase search performance? Discuss the results with your tutor.

\section{Activity 2: Quadratic probing}
As an alternative to linear probing, we can first probe one index along when a collision occurs, then two along, then four, then nine, then sixteen and so on, i.e. the interval between indices that we probe is the square of the number of probes made so far. This can often give better performance, because it reduces primary clustering, i.e. long contiguous chains of displaced elements.

Implement quadratic probing in your hash table. You will need to modify the insert and search functions to do this. You may wish to refactor them to call a function called probe to do the probing, since this code is duplicated between the functions (and in general duplicated code is bad!).

Repeat the previous exercise to measure how the average number of probes scales with the load factor. Do you see an improvement in performance?

\section{Activity 3: Chained hash tables} 
So far, we have assumed that each position in the array can only store one data item, forcing us to resolve collisions by probing through the table for an empty slot (insertion) or until we find the item or NULL (searching). An alternative data structure, known as a chained hash table, resolves collisions by allowing more than one data item to be stored in the same position in the array (bucket). It does this by maintaining a linked list at each bucket. This linked list stores all of the data items that hash to that index in the array. When a collision occurs during insertion, the new data item is added to the \emph{head} of the linked list. Note that the load factor of a chained hash table can thus go above 1. In fact, with a good hash function, the load factor gives the expected number of elements we expect to find in each linked list.

For this activity, you should create your own implementation of a chained hash table, by combining your hash table implementation with your linked list implementation.  

Measure performance for different sizes of table, e.g. insert 10000 data items, but consider tables of size 1000, 2000, 3000 etc up to 10000. Discuss your results with your tutor.


\section{Source code listings}
\subsection{The \texttt{OAHashTable.h} header file}
\begin{lstlisting}
#pragma once

struct DataItem 
{
	int data;
	int key;
};

struct OAHashTable
{
	struct DataItem** buckets;
	int size;
	int numInsertions;
	int numSearches;
	int numProbes;
	int numCollisions;
};

struct OAHashTable* initOAHashTable(int);
void printKeys(struct OAHashTable*);
int insert(struct OAHashTable*, int, int);
void destroy(struct OAHashTable*);
void printPerformance(struct OAHashTable *hashTable);
struct DataItem* search(struct OAHashTable*, int);
\end{lstlisting}

\subsection{The \texttt{OAHashTable.c} implementation file}
\begin{lstlisting}
#include <stdio.h>
#include <stdlib.h>
#include "OAHashTable.h"


int hashCode(int key, int size)
{
    return key % size;
}

int insert(struct OAHashTable *hashTable, int key, int data)
{
    struct DataItem *item = (struct DataItem*) malloc(sizeof(struct DataItem));
    item -> data = data;
    item -> key = key;

    int hashIndex = hashCode(key, hashTable->size);
    int numTries = 0;
    if(hashTable->buckets[hashIndex] != NULL)
    {
        hashTable->numCollisions++;
    }
    while(hashTable->buckets[hashIndex] != NULL && numTries < hashTable->size)
    {
        numTries++;
        ++hashIndex;
        hashIndex %= hashTable->size;
    }
    if(numTries == hashTable->size)
    {
        return -1;
    }
    else
    {
        hashTable->buckets[hashIndex] = item;
    }
    hashTable->numInsertions++;
    return 0;
}

void destroy(struct OAHashTable *hashTable)
{
	for(int i=0; i<hashTable->size; i++)
	{
		if(hashTable->buckets[i] != NULL)
		{
			free(hashTable->buckets[i]);
		}
 	 }
 	 free(hashTable);
 }
 
 struct OAHashTable* initOAHashTable(int size)
 {
 	
 	struct OAHashTable *hashTable = (struct OAHashTable*)malloc(sizeof(struct OAHashTable));
     // Use calloc to initialise the buckets to NULL, so that we can test
     // for empty buckets.
 	hashTable->buckets = (struct DataItem**)calloc(size, sizeof(struct DataItem*));
 	hashTable->size = size;
	hashTable->numCollisions = 0;
	hashTable->numProbes = 0;
	hashTable->numInsertions = 0;
	hashTable->numSearches = 0;
	return hashTable;
 }

void printKeys(struct OAHashTable* hashTable)
{
    printf("[");
    for(int i=0; i<hashTable->size -1; i++)
    {
        if(hashTable->buckets[i] != NULL)
        {
            printf("%d, ", hashTable->buckets[i]->key);
        }
        else
        {
            printf("-, ");
        }
    }
    // Print the last element without a comma and space, just to be neat
    
    if(hashTable->buckets[hashTable->size-1] != NULL)
    {
        printf("%d", hashTable->buckets[hashTable->size-1]->key);
    }
    else
    {
        printf("-");
    }
    printf("]\n");
    
}

void printPerformance(struct OAHashTable *hashTable)
{
    printf("Number of insertions: %d\n", hashTable->numInsertions);
    printf("Number of collisions: %d\n", hashTable->numCollisions);
    printf("Load factor: %.3f\n", (double)hashTable->numInsertions/hashTable->size);
    printf("%% of insertions that collided: %.1f\n", (double)hashTable->numCollisions/hashTable->numInsertions*100);
    printf("Average number of probes per search: %f\n", (double)hashTable->numProbes/hashTable->numSearches);
}

struct DataItem* search(struct OAHashTable* hashTable, int key)
{
    hashTable->numSearches++;
    int hashIndex = hashCode(key,hashTable->size);
    int numTries = 0;
    while(hashTable->buckets[hashIndex] != NULL && numTries < hashTable->size)
    {
        numTries++;
        hashTable->numProbes++;
        if(hashTable->buckets[hashIndex]->key == key)
            return hashTable->buckets[hashIndex];
        
        ++hashIndex;
        hashIndex %= hashTable->size;
    }
    
    return NULL;
}
\end{lstlisting}
\subsection{The \texttt{main.c} file}
\begin{lstlisting}
#include <stdio.h>
#include <stdlib.h>
#include <time.h>
#include "OAHashTable.h"

#define SIZE 11

int main()
{
    struct OAHashTable *hashTable = initOAHashTable(SIZE);
    // Intialise random number generator
    // seeding with the system time
    srand(time(NULL));
    
    // Initialise with given keys and random data
    insert(hashTable,6, rand());
    insert(hashTable,2, rand());
    insert(hashTable,13, rand());
    insert(hashTable,5, rand());
    insert(hashTable,17, rand());
    insert(hashTable,39, rand());
    insert(hashTable,1, rand());
    insert(hashTable,12, rand());
    
    printKeys(hashTable);
    
    search(hashTable, 6);
    search(hashTable, 39);
    search(hashTable, 100);
    
    printPerformance(hashTable);
    
    destroy(hashTable);
    
    return 0;
}
\end{lstlisting}
%\backmatter

\bibliographystyle{plain}

\bibliography{workbook}

\end{document}


%\begin{framed}
%HELLO
%\end{framed}


